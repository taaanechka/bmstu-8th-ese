\chapter{Задания}

\section{Выравнивание загрузки ресурсов в проекте}

Для ликвидации перегрузки ресурсов в проекте будет использоваться автоматическое выравнивание ресурсов.
Результат его применения виден на рисунках \ref{img:task3_1_align_res}-\ref{img:task3_1_align_gant}. На первом из них измененнные поля выделены голубым, на втором --- зеленым цветом показаны задачи до выравнивания, голубым --- после. 
\imgw{task3_1_align_res}{ht!}{0.7\textwidth}{Даты начала и окончания работ}
\imgw{task3_1_align_gant}{ht!}{1\textwidth}{Диаграмма Ганта с выравниванием}

В визуальном оптимизаторе можно увидеть, что ресурсы более не перегружены (рисунок \ref{img:task3_1_align_view_opt}).
\imgw{task3_1_align_view_opt}{ht!}{1\textwidth}{Визуальный оптимизатор}

\subsection*{Вывод}
Таким образом, перегрузка ресурсов была ликвидирована посредством их автоматического выравнивания, а дата окончания проекта сдвинулась с 15.09.23 на \textbf{19.09.23}. Стоимость проекта осталась прежней (в пределах выделенного бюджета), а срок завершения проекта так же до сих пор превышен.

\section{Учет периодических задач в плане проекта}

Создание повторяющейся еженедельной задачи <<Совещание>> по 
средам с 10 до 11 утра приведено на рисунке \ref{img:task3_2_ret_task_plan}.
\imgw{task3_2_ret_task_plan}{ht!}{0.9\textwidth}{Создание повторяющейся еженедельной задачи <<Совещание>>}

Результат создания повторяющейся еженедельной задачи <<Совещание>> приведен на рисунке \ref{img:task3_2_ret_tasks}.
\imgw{task3_2_ret_tasks}{ht!}{0.8\textwidth}{Повторяющаяся еженедельная задача <<Совещание>>}

На повторяющуюся задачу назначены все  специалисты, кроме 
наборщиков данных и программистов №1-4, т.к. их интересы на совещании 
представляет ведущий программист (рисунок \ref{img:task3_2_ret_task_resourses}).
\imgw{task3_2_ret_task_resourses}{ht!}{0.8\textwidth}{Назначение ресурсов на задачу <<Совещание>>}

Из русунка \ref{img:task3_2_ret_tasks_cost} видно, что после назначения на задачу <<Совещание>> ресурсы стали перегружены и стоимость данной повторяющейся задачи составила \textbf{20 039 рублей}, из-за чего стоимость всего проекта выросла на стоимость проведения совещаний и составила \textbf{68 085 рублей} (рисунок \ref{img:task3_2_gen_cost_0}).
\imgw{task3_2_ret_tasks_cost}{ht!}{0.9\textwidth}{Стоимость задачи <<Совещание>> до оптимизации затрат за нее}
\imgw{task3_2_gen_cost_0}{ht!}{0.9\textwidth}{Стоимость проекта до оптимизации затрат на задачу <<Совещание>>}

На рисунках \ref{img:task3_2_ret_view_1}-\ref{img:task3_2_ret_view_2} представлены вышеупомянутые перегрузки ресурсов на конкретных задачах. Все эти перегрузки возникли из-за наложения проведения совещаний на выполняемые сотрудниками другие задачи.
\imgw{task3_2_ret_view_1}{ht!}{0.9\textwidth}{Перегрузки ресурсов из-за <<Совещания>> (часть 1)}
\imgw{task3_2_ret_view_2}{ht!}{0.9\textwidth}{Перегрузки ресурсов из-за <<Совещания>> (часть 2)}

Необходимо ликвидировать возникшую перегрузку ресурсов посредством автоматического выравнивания и дополнительно прервать задачи, на которых ресурсы остались перегружены. Также требуется ликвидировать превышение бюджета. Для этого для всех ресурсов, назначенных на повторяющуюся задачу <<Совещание>>, был составлен план затрат B без учета затрат на использование (рисунок \ref{img:task3_2_cost_plan_b}).
\imgw{task3_2_cost_plan_b}{ht!}{0.7\textwidth}{план затрат B}

Затем для всех ресурсов, назначенных на повторяющуюся задачу <<Совещание>>, был выбран \textbf{план затрат B} в таблице норм затрат (рисунок \ref{img:task3_2_cost_plan_resourses}).
\imgw{task3_2_cost_plan_resourses}{ht!}{0.7\textwidth}{Назначение ресурсам <<Совещания>> плана затрат B}

После устранения перегрузки ресурсов и изменения плана затрат стоимость совещания уменьшилась с 20 039 до \textbf{1 769 рублей} (рисунок \ref{img:task3_2_meeting_cost}), а стоимость проекта уменьшилась с 68 085 до \textbf{48 849 рублей} (рисунок \ref{img:task3_2_gen_cost_1}).
\imgw{task3_2_meeting_cost}{ht!}{0.9\textwidth}{Стоимость задачи <<Совещание>> после оптимизации затрат}
\imgw{task3_2_gen_cost_1}{ht!}{0.9\textwidth}{Стоимость проекта после оптимизации затрат на <<Совещание>>}

\subsection*{Вывод}
Таким образом, с выравниванием ресурсов после добавления периодической задачи \textbf{<<Совещание>>} дата окончания проекта сдвинулась с 19.09.23 до \textbf{25.09.23} (все еще за пределами сроков, нуждается в оптимизации), а стоимость проекта с проведением совещаний увеличилась на 1 769 рублей и составила \textbf{49 849 рублей}, но осталась в пределах бюджета.

\section{Оптимизация критического пути}

\subsection{Критический путь и его оптимизация}

Среди задач, лежащих на критическом пути (рисунок \ref{img:task3_3_critical_sorted_by_duration}), наибольшую длительность имеют задачи, связанные с программированием, следовательно, они оказывают наибольшее влияние на срок 
реализации проекта.
\imgw{task3_3_critical_sorted_by_duration}{ht!}{1\textwidth}{Задачи критического пути, отсортированные по продолжительности}

В визуальном оптимизаторе видно, что программисты распределены на задачи, связанные с программированием, неравномерно (рисунок \ref{img:task3_3_view_program_before}). 
\imgw{task3_3_view_program_before}{ht!}{1\textwidth}{Начальное распределение программистов на их задачи}

Следовательно, можно перераспределить эти ресурсы между задачами программирования более равномерно (рисунок \ref{img:task3_3_view_program_after}), что способствует ускорению завершения этих задач и всего проекта. \textit{После этого следует удалить все совещания позже даты последней задачи проекта.} Для равномерного распределения программистов по их задачам были изменены их назначения на следующие задачи:
\begin{itemize}[label = ---]
	\item <<14. Создание модели ядра>> (программисты №1-4);
	\item <<10. Программирование средств обработки>> (программисты №2-4);
	\item <<16. Создание рабочей версии ядра>> (программисты №1-4);
	\item <<7. Программирование интерфейса>> (программисты №1-4);
	\item <<26. Тестирование сайта>> (программисты №1-4).
\end{itemize}
\imgw{task3_3_view_program_after}{ht!}{1\textwidth}{Равномерное распределение программистов на их задачи}

После перераспределения программистов на их задачи новых перегрузок не возникло, что также видно из визуального оптимизатора (рисунок \ref{img:task3_3_view_program_after}).

Исходя из результатов второй лабороторной работы, программисты являются \textbf{высокооплачиваемыми} специалистами, а значит, сокращение времени их работы позволит существенно \textbf{уменьшить затраты} на проект. Как видно из рисунка \ref{img:task3_3_gen_cost_gant}, дата окончания проекта сдвинулась с 25.09.23 на \textbf{20.07.23} (продолжительность всего проекта сократилась почти на 2 месяца и теперь не превышает установленного срока), а стоимость проекта уменьшилась с 49 849 до \textbf{48 480,18 рублей} (уменьшилась на 1 368,82 рублей и не превышает бюджета проекта).
\imgw{task3_3_gen_cost_gant}{ht!}{1\textwidth}{Стоимость проекта и диаграмма Ганта после оптимизации}

\subsection{Анализ затрат по группам ресурсов}

Информация о трудозатратах по группам ресурсов представлена в 
графическом виде для проекта на моменты его конечных состояний из ЛР №2 и ЛР №3 (рисунки \ref{img:task2_3_group_laborcost}-\ref{img:task3_3_group_laborcost}).

\imgw{task2_3_group_laborcost}{ht!}{0.7\textwidth}{Информация о трудозатратах (ЛР №2)}
\imgw{task3_3_group_laborcost}{ht!}{0.7\textwidth}{Информация о трудозатратах (ЛР №3)}

\newpage
По результатам добавления в план проекта совещаний, ликвидации перегрузки ресурсов, оптимизации затрат и критического пути трудозатраты изменились для следующих групп:
\begin{itemize}[label = ---]
	\item Нет назначения (сервер) --- уменьшились на 2\%;
	\item Анализ --- увеличились на 1\%;
	\item Мульти-медиа --- увеличились на 1\%.
\end{itemize}

Информация о затратах по группам ресурсов представлена для проекта на моменты его конечных состояний из ЛР №2 и ЛР №3 (рисунки \ref{img:task2_3_group_cost}-\ref{img:task3_3_group_cost}).

\imgw{task2_3_group_cost}{ht!}{0.7\textwidth}{Информация о затратах (ЛР №2)}
\imgw{task3_3_group_cost}{ht!}{0.7\textwidth}{Информация о затратах (ЛР №3)}

\newpage
По результатам добавления в план проекта совещаний, ликвидации перегрузки ресурсов, оптимизации затрат и критического пути затраты изменились для следующих групп:
\begin{itemize}[label = ---]
	\item Нет назначения (сервер) --- уменьшились на 1\%;
	\item Программирование --- уменьшились на 1\%;
	\item Анализ --- увеличились на 1\%;
	\item Документация --- увеличились на 1\%.
\end{itemize}

\newpage
По результатам выполнения \textbf{ЛР №2} получились следующие соотношения \textbf{<<Затраты---Трудозатраты>>}:
\begin{itemize}[label = ---]
	\item Анализ --- $\frac{10}{2}=5$;
	\item Программирование --- $\frac{50}{29}=1.72$;
	\item Нет назначения (сервер) --- $\frac{13}{32}=0.4$;
	\item Ввод данных --- $\frac{11}{26}=0.42$;
	\item Документация --- $\frac{2}{2}=1$.
\end{itemize}

По результатам выполнения \textbf{ЛР №3} получились следующие соотношения \textbf{<<Затраты---Трудозатраты>>}:
\begin{itemize}[label = ---]
	\item Анализ --- $\frac{11}{3}=3.67$ (уменьшилось);
	\item Программирование --- $\frac{49}{29}=1.69$ (уменьшилось);
	\item Нет назначения (сервер) --- $\frac{12}{30}=0.4$;
	\item Ввод данных --- $\frac{11}{26}=0.42$;
	\item Документация --- $\frac{3}{2}=1.5$ (увеличилось).
\end{itemize}

Сохранение \textbf{базового плана} проекта приведено на рисунке \ref{img:task3_3_base_plan}. \imgw{task3_3_base_plan}{ht!}{0.5\textwidth}{Базовый план проекта}

\subsection*{Вывод}
Таким образом, соотношение \textbf{<<Затраты---Трудозатраты>>} уменьшилось для высокооплачиваемых специалистов (программистов) и самых высокооплачиваемых специалистов (аналитики), но увеличилось для низкооплачиваемого специалиста (технический писатель), что в результате привело к уменьшению стоимости проекта.

\section{Вывод по работе}

По результатам выполнения ЛР №2 программисты являются высокооплачиваемыми специалистами (на них приходится 50\% затрат и лишь 29\% трудозатрат проекта), значит, при сокращении времени их работы удастся сократить затраты проекта. В процессе выполнения ЛР №3 было выяснено, для оптимизации критического пути следует сократить время на задачи программирования. Т.к. от продолжительности этих задач зависит время начала следующих, сократив их продолжительность, удалось сократить общее время выполнения проекта. Это было реализовано посредством более равномерного перераспределения программистов на их задачи.

После проведенных оптимизаций также удалось уменьшить соотношение \textbf{<<Затраты---Трудозатраты>>} для программистов (высокооплачиваемые) и аналитиков (самые высокооплачиваемые), при этом оно увеличилось для технического писателя (низкооплачиваемый). В результате удалось уменьшить затраты и продолжительность проекта.
Дата окончания проекта сдвинулась с 25.09.23 на \textbf{20.07.23} (продолжительность всего проекта сократилась почти на 2 месяца и теперь не превышает установленного срока), а стоимость проекта уменьшилась с 49 849 до \textbf{48 480,18 рублей} (уменьшилась на 1 368,82 рублей и не превышает бюджета проекта).

