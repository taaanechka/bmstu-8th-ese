\chapter{Вступление из лабороторной №2}

\textbf{Краткое описание проекта}. Команда разработчиков из 16 человек занимается созданием карты
города на основе собственного модуля отображения. Проект должен быть завершен в течение
\textbf{6 месяцев}. 

Бюджет проекта: \textbf{50 000 рублей}.

\section{Ресурсы и причины их перегрузки}

Ниже приведена краткая информация о ресурсах (рисунок \ref{img:task_3_group}).
\imgw{task_3_group}{ht!}{1\textwidth}{Ресурсный лист}

После назначения ресурсов задачам видно, что возникают перегрузки относительно ресурсов, каждый из которых
одновременно занят какими-либо двумя задачами (рисунки \ref{img:task_2_view_1}-\ref{img:task_2_view_2}).
Так перегрузки возникают у \textbf{системного аналитика} во время выполнения задач <<Анализ и построение структуры базы объектов>> и <<Анализ и проектирование ядра>>, у \textbf{художника-дизайнера} во время выполнения задач <<Разработка дизайна руководства>> и <<Разработка дизайна сайта>>, у \textbf{технического писателя} во время выполнения задач <<Написание руководства пользователя>> и <<Создание справочной системы>>.
\imgw{task_2_view_1}{ht!}{1\textwidth}{Перегрузка ресурсов (часть 1)}
\imgw{task_2_view_2}{ht!}{1\textwidth}{Перегрузка ресурсов (часть 2)}

\section{Способы устранения перегрузок ресурсов}

Для устранения перегрузки ресурсов существуют следующие спососбы:
\begin{itemize}[label = ---]
	\item изменить календарь работы ресурса;
	\item назначить ресурс на неполный рабочий день;
	\item изменить профиль назначения ресурса;
	\item изменить ставку оплаты ресурса;
	\item добавить ресурсу время задержки;
	\item разбить задачу на этапы и перекрыть по времени их выполнение;
	\item применить автоматическое выравнивание.
\end{itemize}
