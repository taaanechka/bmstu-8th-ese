\chapter{Задания}

\section{Задание 1}

Исследовать влияние атрибутов персонала (ACAP, PCAP, AEXP, LEXP) на трудоемкость (РМ) и время разработки (ТМ) для модели
COCOMO. Для этого, взяв за основу любой из типов проекта (обычный, встроенный или промежуточный), получить значения PM и 
ТМ для одного и того же значения параметра SIZE (размера программного кода), выбрав номинальный (средний) уровень 
сложности продукта (CPLX) и изменяя значения характеристик 
персонала от очень низких до очень высоких.  Повторить расчеты для 
проекта, предусматривающего создание продукта очень низкой и очень 
высокой сложности.

Атрибуты персонала:
\begin{itemize}[label = ---]
	\item ACAP --- способности аналитика;
	\item AEXP --- знание приложений;
	\item PCAP --- способности программиста;
	\item LEXP --- знание языка программирования.
\end{itemize}

Возьмем SIZE = 100 KLOC. Тогда графики зависимостей трудоемкости и времени разработки от атрибутов персонала 
для номинальной сложности продукта (CPLX) представлены на рисунке~\ref{img:gr_2}.
\imgw{gr_2}{ht!}{0.8\textwidth}{Зависимости TM и PM от атрибутов персонала при номинальной CPLX}

Графики зависимостей трудоемкости и времени разработки от атрибутов персонала 
для очень низкой сложности продукта (CPLX) представлены на рисунке~\ref{img:gr_0}.
\imgw{gr_0}{ht!}{0.8\textwidth}{Зависимости TM и PM от атрибутов персонала при очень низкой CPLX}

Графики зависимостей трудоемкости и времени разработки от атрибутов персонала 
для очень высокой сложности продукта (CPLX) представлены на рисунке~\ref{img:gr_4}.
\imgw{gr_4}{ht!}{0.8\textwidth}{Зависимости TM и PM от атрибутов персонала при очень высокой CPLX}

С ростом значений атрибутов персонала происходит уменьшение трудоемкости и времени разработки, как при номинальном уровне сложности продукта, 
так и при очень низком и при очень высоком уровне сложности продукта. Таким образом, для всех уровней сложности продукта 
повышение квалификационных характеристик сотрудников приводит к снижению трудозатрат и времени разработки. 

Если сравнивать графики разных режимов, можно заметить, что для всех уровней сложности продукта трудоемкость является максимальной 
при встроенном режиме, а время разработки --- при обычном режиме, и наоборот: трудоемкость минимальна при обычном режиме, а 
время разработки --- при встроенном режиме. Также можно сказать,
что на сроки реализации наиболее влияющим фактором на высокой сложности проекта оказываются способности аналитика и программиста. 

Если сравнивать персонал между собой, 
из графиков видно, что на время разработки наибольшее влияние оказывают способности аналитика и способности программиста для всех уровней сложности продукта. Исходя из этого, для сокращения периода реализации проекта выгоднее повышать квалификационные характеристики аналитика и программиста.

\section{Задание 2}

Для выполнения данного задания проект был расчитан по следующим параметрам:
\begin{itemize}[label = ---]
	\item SIZE (размер) = 25 KLOC;
	\item PCAP (способности программиста) --- высокий;
	\item LEXP (знание языка программирования) --- высокий;
	\item MODP (использование современных методов) --- очень высокий;
	\item TOOL (использование программных инструментов) --- высокий;
	\item Режим проекта --- обычный.
\end{itemize}

\imgw{t2_work}{ht!}{0.9\textwidth}{Расчет проекта}

\imgw{t2}{ht!}{0.7\textwidth}{Диаграмма привлечения сотрудников}

В пиковом значении диаграммы (рисунок~\ref{img:t2}) одновременно задействовано 8 сотрудников.

\newpage
Произведем расчет бюджета по диаграмме привлечения сотрудников (рисунок~\ref{img:t2}) с использованием медианной месячной
зарплаты за второе полугодие 2022 года по москве по данным сервиса Хабр
Карьера:
\begin{itemize}[label = ---]
	\item Системный аналитик – 160~000 рублей;
	\item Разработчик – 200~000 рублей;
	\item Менеджер продукта – 230~000 рублей;
	\item Инженер по тестированию – 150~000 рублей.
\end{itemize}

Расчет бюджета в соответствии с планом:
\begin{itemize}[label = ---]
	\item Планирование и определение требований (Менеджер продукта) – 920~000 рублей;
	\item Проектирование продукта (Системный аналитик + Менеджер продукта)
	– 1~560~000 рублей;
	\item Детальное проектирование (Системный аналитик X2 + Менеджер продукта + Разработчик X4) – 2~700~000 рублей;
	\item Кодирование и тестирование отдельных модулей (Разработчик Х4
	+ Инженер по тестированию Х4) – 2~800~000 рублей;
	\item Интеграция и тестирование (Разработчик Х3 + Инженер по тестированию Х3) – 3~150~000 рублей.
\end{itemize}

Итоговая \textbf{стоимость} проекта: 11 130 000 рублей.

\textbf{Трудоемкость} проекта составила 61.87 человеко-месяцев, а \textbf{время разработки} --- 15.83 месяцев.

\section*{Выводы}

Методика COCOMO является подходящей для предварительной оценки длительности и стоимости проекта на каждом из основных этапов. Однако для более
детального планирования проекта следует использовать другие средства,
позволяющие учитывать затраты и длительность более подробно, а также
позволяющие предусматривать другие параметры проекта.


