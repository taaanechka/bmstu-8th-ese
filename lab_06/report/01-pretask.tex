\chapter{Описание COCOMO}

\textbf{Вариант №1}

\textbf{COnstructive COst MOdel} --- алгоритмическая модель оценки стоимости разработки программного обеспечения, разработанная Барри Боэмом.
Модель использует простую формулу регрессии с параметрами, определенными из данных, собранных по ряду проектов.
$$PM = C1 \cdot EAF \cdot {SIZE}^{p1}$$
$$TM = C2 \cdot {PM}^{p2}$$

PM (трудозатраты) --- количество человеко-месяцев.

C1 --- масштабирующий коэффициент.

EAF --- уточняющий фактор, характеризующий предметную область, персонал, среду и инструментарий, используемый для создания рабочих продуктов процесса.

SIZE (размер) --- размер конечного продукта (кода, созданного человеком), измеряемый в исходных инструкциях (DSI, delivered source instructions), которые необходимы для реализации требуемой функциональной возможности.

P1 --- показатель степени, характеризующий экономию при больших масштабах, присущую тому процессу, который используется для создания конечного продукта; в частности, способность процесса избегать непроизводительных видов деятельности (доработок, бюрократических проволочек,
накладных расходов на взаимодействие).

TM (время) --- общее количество месяцев.

С2 --- масштабирующий коэффициент для сроков исполнения.

Р2 --- показатель степени, который характеризует инерцию и распараллеливание, присущие управлению разработкой ПО.

%Бюджет проекта: \textbf{50 000 рублей}.
%
%\section{Ресурсы и причины их перегрузки}
%
%Ниже приведена краткая информация о ресурсах (рисунок~\ref{img:task_3_group}).
%\imgw{task_3_group}{ht!}{1\textwidth}{Ресурсный лист}
%
%После назначения ресурсов задачам видно, что возникают перегрузки относительно ресурсов, каждый из которых
%одновременно занят какими-либо двумя задачами (рисунки~\ref{img:task_2_view_1}-\ref{img:task_2_view_2}).
%Так перегрузки возникают у \textbf{системного аналитика} во время выполнения задач <<Анализ и построение структуры базы объектов>> и <<Анализ и проектирование ядра>>, у \textbf{художника-дизайнера} во время выполнения задач <<Разработка дизайна руководства>> и <<Разработка дизайна сайта>>, у \textbf{технического писателя} во время выполнения задач <<Написание руководства пользователя>> и <<Создание справочной системы>>.
%\imgw{task_2_view_1}{ht!}{1\textwidth}{Перегрузка ресурсов (часть 1)}
%\imgw{task_2_view_2}{ht!}{1\textwidth}{Перегрузка ресурсов (часть 2)}
%
%\section{Способы устранения перегрузок ресурсов}
%
%Для устранения перегрузки ресурсов существуют следующие спососбы:
%\begin{itemize}[label = ---]
%	\item изменить календарь работы ресурса;
%	\item назначить ресурс на неполный рабочий день;
%	\item изменить профиль назначения ресурса;
%	\item изменить ставку оплаты ресурса;
%	\item добавить ресурсу время задержки;
%	\item разбить задачу на этапы и перекрыть по времени их выполнение;
%	\item применить автоматическое выравнивание.
%\end{itemize}
%
%\section{Выравнивание загрузки ресурсов в проекте}
%
%Для ликвидации перегрузки ресурсов в проекте используется автоматическое выравнивание ресурсов.
%В результате его применения перегрузка ресурсов была ликвидирована, а дата окончания проекта сдвинулась с 15.09.23 на \textbf{19.09.23}. Стоимость проекта осталась прежней (в пределах выделенного бюджета), а срок завершения проекта так же до сих пор превышен.
%
%\imgw{task3_1_align_res}{ht!}{0.7\textwidth}{Даты начала и окончания работ}
%\imgw{task3_1_align_gant}{ht!}{1\textwidth}{Диаграмма Ганта с выравниванием}
%
%В визуальном оптимизаторе можно увидеть, что ресурсы более не перегружены (рисунок~\ref{img:task3_1_align_view_opt}).
%\imgw{task3_1_align_view_opt}{ht!}{1\textwidth}{Визуальный оптимизатор}
%
%\section{Учет периодических задач в плане проекта}
%
%Создана еженедельная задача <<Совещание>>, повторяющаяся по средам с 10 до 11 утра.Результат создания повторяющейся еженедельной задачи <<Совещание>> приведен на рисунке \ref{img:task3_2_ret_tasks}.
%\imgw{task3_2_ret_tasks}{ht!}{0.8\textwidth}{Повторяющаяся еженедельная задача <<Совещание>>}
%
%На повторяющуюся задачу назначены все  специалисты, кроме 
%наборщиков данных и программистов №1-4, т.к. их интересы на совещании 
%представляет ведущий программист.
%
%Возникшая перегрузка ресурсов была ликвидирована посредством автоматического выравнивания и дополнительно прерваны задачи, на которых ресурсы остались перегружены. Для ликвидации превышения бюджета для всех ресурсов, назначенных на повторяющуюся задачу <<Совещание>>, был составлен и назначен план затрат B без учета затрат на использование.
%
%После устранения перегрузки ресурсов и изменения плана затрат стоимость совещания уменьшилась с 20 039 до \textbf{1 769 рублей} (рисунок~\ref{img:task3_2_meeting_cost}), а стоимость проекта уменьшилась с 68 085 до \textbf{48 849 рублей} (рисунок~\ref{img:task3_2_gen_cost_1}).
%\imgw{task3_2_meeting_cost}{ht!}{0.9\textwidth}{Стоимость задачи <<Совещание>> после оптимизации затрат}
%\imgw{task3_2_gen_cost_1}{ht!}{0.9\textwidth}{Стоимость проекта после оптимизации затрат на <<Совещание>>}
%
%\section{Оптимизация критического пути}
%
%\subsection{Критический путь и его оптимизация}
%
%Среди задач, лежащих на критическом пути (рисунок~\ref{img:task3_3_critical_sorted_by_duration}), наибольшую длительность имеют задачи, связанные с программированием, следовательно, они оказывают наибольшее влияние на срок 
%реализации проекта.
%\imgw{task3_3_critical_sorted_by_duration}{ht!}{1\textwidth}{Задачи критического пути, отсортированные по продолжительности}
%
%В визуальном оптимизаторе видно, что программисты распределены на задачи, связанные с программированием, неравномерно (рисунок~\ref{img:task3_3_view_program_before}). 
%\imgw{task3_3_view_program_before}{ht!}{1\textwidth}{Начальное распределение программистов на их задачи}
%
%Следовательно, можно перераспределить эти ресурсы между задачами программирования более равномерно (рисунок~\ref{img:task3_3_view_program_after}), что способствует ускорению завершения этих задач и всего проекта.
%\imgw{task3_3_view_program_after}{ht!}{1\textwidth}{Равномерное распределение программистов на их задачи}
%
%\newpage
%Как видно из рисунка \ref{img:task3_3_gen_cost_gant}, дата окончания проекта сдвинулась с 25.09.23 на \textbf{20.07.23} (продолжительность всего проекта сократилась почти на 2 месяца и теперь не превышает установленного срока), а стоимость проекта уменьшилась с 49 849 до \textbf{48 480,18 рублей} (уменьшилась на 1 368,82 рублей и не превышает бюджета проекта).
%\imgw{task3_3_gen_cost_gant}{ht!}{1\textwidth}{Стоимость проекта и диаграмма Ганта после оптимизации}
