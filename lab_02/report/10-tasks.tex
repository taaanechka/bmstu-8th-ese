\chapter{Задания}

\textbf{Краткое описание проекта}. Команда разработчиков из 16 человек занимается созданием карты
города на основе собственного модуля отображения. Проект должен быть завершен в течение
6 месяцев. 

Бюджет проекта: 50 000 рублей.

\section{Создание списка ресурсов}

Ресурсный лист заполнен в соответствии с таблицей (рисунок \ref{img:task_1_resources}).
\imgw{task_1_resources}{ht!}{1\textwidth}{Ресурсный лист}

\section{Назначение ресурсов задачам}

Ресурсы назначены задачам 
в соответствии с таблицей (рисунки \ref{img:task_2_gant_1}-\ref{img:task_2_gant_2}).
\imgw{task_2_gant_1}{ht!}{1\textwidth}{Назначение ресурсов задачам (часть 1)}
\imgw{task_2_gant_2}{ht!}{1\textwidth}{Назначение ресурсов задачам (часть 2)}

В визуальном оптимизаторе видно, что возникают перегрузки относительно ресурсов, каждый из которых
одновременно занят какими-либо двумя задачами (рисунки \ref{img:task_2_view_1}-\ref{img:task_2_view_2}).
Так перегрузки возникают у \textbf{системного аналитика} во время выполнения задач <<Анализ и построение структуры базы объектов>> и <<Анализ и проектирование ядра>>, у \textbf{художника-дизайнера} во время выполнения задач <<Разработка дизайна руководства>> и <<Разработка дизайна сайта>>, у \textbf{технического писателя} во время выполнения задач <<Написание руководства пользователя>> и <<Создание справочной системы>>.
\imgw{task_2_view_1}{ht!}{1\textwidth}{Перегрузка ресурсов (часть 1)}
\imgw{task_2_view_2}{ht!}{1\textwidth}{Перегрузка ресурсов (часть 2)}

Задачам 2, 8 и 12 заданы фиксированные траты по 1000 р. (рисунок \ref{img:task_2_fixed_cost}).
\imgw{task_2_fixed_cost}{ht!}{0.5\textwidth}{Фиксированные траты}

Для задачи № 8 «Построение базы объектов» арендован дополнительный 
сервер со стоимостью аренды --- 2 рубля в час
(рисунок \ref{img:task_2_server_cost}).
\imgw{task_2_server_cost}{ht!}{1\textwidth}{Добавление сервера в ресурсы}
%\imgw{task_2_new_cost}{ht!}{1\textwidth}{Фиксированные траты после добавления сервера}

После добавления сервера увеличилась стоимость как 8 задачи, так и всего проекта (рисунок \ref{img:task_2_end}).
\imgw{task_2_end}{ht!}{1\textwidth}{Стоимость проекта}

\clearpage
\section{Анализ затрат по группам ресурсов}
Проведена структуризация затрат по группам ресурсов (рисунок \ref{img:task_3_group}).
\imgw{task_3_group}{ht!}{1\textwidth}{Структуризация затрат по группам ресурсов}

Информация о затратах и трудозатратах по структурным группам ресурсов представлена в 
графическом виде (рисунок \ref{img:task_3_group_cost}).
\imgw{task_3_group_cost}{ht!}{1\textwidth}{Информация о затратах}

\newpage
Трудозатраты на \textit{программирование} и \textit{ввод данных} примерно равны, но при этом затраты на \textit{программирование}
почти в 5 раз больше, чем на \textit{ввод данных}. Причем коэффициент, равный отношению процента затрат к 
проценту трудозатрат, для \textit{программирования} --- 1.7, для \textit{ввода данных} --- 0.4, для  \textit{анализа} --- 5, для \textit{сервера} --- 0.4.

\section{Вывод}

Аренда сервера составляет 13\% затрат и 32\% трудозатрат проекта.

Программисты, аналитики и группа ввода данных являются дорогостоящими специалистами.

С помощью визуального оптимизатора удалось определить перегрузки. Они возникают в
определенные моменты времени относительно тех ресурсов, каждый из которых 
занят в эти моменты какими-либо двумя задачами.

\textbf{Трудозатраты} проекта составили \textbf{9377 часов}. \textbf{Затраты} проекта составили \textbf{48 094 рублей}, что \textit{не превышает} выделенного на проект бюджета.
