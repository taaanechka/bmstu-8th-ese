\chapter{Задание для тренировки}

\textbf{Вариант №0.}

\begin{enumerate}
    \item Дополнить временной план проекта, подготовленный на предыдущем этапе
    (лабораторная работа № 1), информацией о ресурсах и определить стоимость 
    проекта.
    \item Для этого заполнить ресурсный лист в программе MS Project, принимая во
    внимание, что к реализации проекта привлекается не более 10 человек.
    \item Предусмотреть, что стандартная ставка ресурса составляет 150 руб./день.
    \item Произвести назначение ресурсов на задачи в соответствии с таблицей. С учетом 
    того, что квалификация ресурсов одинаковая, при назначении ресурсов 
    использовать процент загрузки.
    \imgw{screen_training_2}{ht!}{0.5\textwidth}{}
    \item Запланировать для выполнения работы Е использование материального ресурса, 
    стоимостью 300 руб. за единицу и с нормой расхода 4 единицы в день.
\end{enumerate}

На вкладке \textbf{Вид (представления ресурсов) $ \rightarrow $ Лист ресурсов} заполнен ресурсный лист по тренировочному заданию (рисунок \ref{img:training_resources}).
\imgw{training_resources}{ht!}{1\textwidth}{Ресурсный лист тренировочного задания}

На вкладке \textbf{Вид (представления задач) $ \rightarrow $ Диаграмма Ганта} представлена диаграмма Ганта тренировочного задания с назначением ресурсов (рисунок \ref{img:training_resources_gant_1}).
\imgw{training_resources_gant_1}{ht!}{1\textwidth}{Диаграмма Ганта тренировочного задания с назначением ресурсов}

\textbf{Вид (данные) $ \rightarrow $ Таблицы $ \rightarrow $ Затраты} (рисунок \ref{img:training_sum_cost_1}): общий бюджет проекта составил 32 100 рублей.
\imgw{training_sum_cost_1}{ht!}{0.7\textwidth}{Бюджет проекта тренировочного задания}

На вкладке \textbf{Вид (представления ресурсов) $ \rightarrow $ Визуальный оптимизатор} (рисунок \ref{img:training_virt_opt}) видно, что в некоторые моменты времени рабочие были перегружены, т.к. их задачи накладывались друг на друга.
\imgw{training_virt_opt}{ht!}{1\textwidth}{Визуальный оптимизатор}
