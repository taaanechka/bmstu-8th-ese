\chapter{Задание для тренировки}

\textbf{Вариант №0.}

Осуществить планирование проекта со следующими временными характеристиками:
\imgw{screen_training}{ht!}{0.5\textwidth}{}

Дата начала проекта --- 1-й рабочий день февраля текущего года.
Провести планирование работ проекта, учитывая следующие связи между задачами:
\begin{itemize}[label = ---]
    \item предусмотреть, что A, E и F – исходные работы проекта, которые можно начинать 
    одновременно;
    \item работы B и I начинаются сразу по окончании работы F;
    \item работа J следует за E, а работа C – за A;
    \item работы H и D следуют за B, но не могут начаться, пока не завершена C;
    \item работа G начинается после завершения H и J.
\end{itemize}

Параметры рабочей среды MS Project, которые использовались при создании плана пробного проекта, 
были стандартными (рисунок \ref{img:training_params}).
\imgw{training_params}{ht!}{1\textwidth}{Стандартные параметры рабочей среды MS Project}
Длительность проекта --- 28 дней (начало:~01.02.2023; окончание:~15.03.2023).

\newpage
Диаграмма Ганта приведена на рисунке \ref{img:training_gantt}.
\imgw{training_gantt}{ht!}{1\textwidth}{Диаграмма Ганта тренировочного задания}