\chapter{Задания}

\textbf{Краткое описание проекта}. Команда разработчиков из 16 человек занимается созданием карты
города на основе собственного модуля отображения. Проект должен быть завершен в течение
6 месяцев. 

Бюджет проекта: 50 000 рублей.

\section{Настройка рабочей среды проекта}

\textbf{Проект $ \rightarrow $ Сведения о проекте}: установлена дата начала проекта.
\imgw{task_1_datefrom}{ht!}{0.8\textwidth}{Установка даты начала проекта}

\textbf{Файл $ \rightarrow $ Параметры}: установлены длительность работы (недели), объем работ (часы), тип работ (с фиксированными трудозатратами); 8 рабочих ч/дн, 40 рабочих ч/нед; начало рабочей недели (понедельник) и финансового года (январь). Продолжительность рабочего дня --- с 9 до 18 часов.
\imgw{task_1_params}{ht!}{0.8\textwidth}{Параметры проекта}

\clearpage
На вкладке \textbf{Проект $ \rightarrow $ Изменение рабочего времени} установлен стандартный календарь рабочего времени 
(рисунок \ref{img:task_1_worktime}) и отмечены праздники (рисунок \ref{img:task_1_holidays}).
\imgw{task_1_worktime}{ht!}{0.8\textwidth}{Рабочее время}
\imgw{task_1_holidays}{ht!}{0.8\textwidth}{Праздники}

Выведена на экран суммарная задача проекта (рисунок \ref{img:task_1_sumtask}) и заполнена вкладка 
\textbf{Задача $ \rightarrow $ Суммарная задача $ \rightarrow $ Заметки} (рисунок \ref{img:task_1_note})
информацией об основных параметрах проекта (его длительности, бюджете и количественном составе команды).
\imgw{task_1_sumtask}{ht!}{0.35\textwidth}{Суммарная задача проекта}
\imgw{task_1_note}{ht!}{1\textwidth}{Заметка}

\section{Создание списка задач}

Составлен список задач в соотвествии с заданием (рисунок \ref{img:task_2_task_list}).
\imgw{task_2_task_list}{ht!}{1\textwidth}{Список задач}

\clearpage
\section{Структурирование списка задач}

Структурирован список задач в соотвествии с заданием (рисунок \ref{img:task_3_group_list}).
\imgw{task_3_group_list}{ht!}{1\textwidth}{Структурированный список задач}

\clearpage
\section{Установление связей между задачами}

Установлены связи между задачами в соотвествии с заданием (рисунки \ref{img:task_4_links_list}~---~\ref{img:task_4_links_list}).
\imgw{task_4_links_list}{ht!}{1\textwidth}{Установление связей между задачами (часть 1)}
\imgw{task_4_links_list_2}{ht!}{1\textwidth}{Установление связей между задачами (часть 2)}

\section{Вывод}

В данной лабораторной работе были освоены возможности программы Microsoft
Project для планирования проекта по разработке программного обеспечения. Составленный план проекта 
позволил получить дату окончания работ --- 15.09.2023. Таким образом, уже на этапе планирования работ
стало известно о превышении сроков проекта на 15 дней.