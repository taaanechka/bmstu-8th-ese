\chapter{Задания}

\textbf{Вариант №1}

\textbf{Масштаб проекта}: группа из 7 чел., длительность не более 3 мес., бюджет не более 650 тыс. руб.

\section{Настройка рабочей среды}

Установлена дата начала проекта --- первый понедельник 2-го квартала текущего года --- 03.04.23 (рисунок~\ref{img:rk_date_start}).
\imgw{rk_date_start}{ht!}{0.7\textwidth}{Дата начала проекта}

Установлены параметры расписания проекта (рисунок~\ref{img:rk_params}).
\imgw{rk_params}{ht!}{0.7\textwidth}{Параметры расписания проекта}

Установлен стандартный календарь и учтены праздники (рисунок~\ref{img:rk_holidays}).
\imgw{rk_holidays}{ht!}{0.7\textwidth}{Праздники}

\section{Создание списка задач}

Создан список задач в соответствии с таблицей (рисунок~\ref{img:rk_tasks_list}).
\imgw{rk_tasks_list}{ht!}{0.7\textwidth}{Список задач}

\section{Структурирование списка задач}

Задачи 5-6 сгруппированы как подзадачи задачи 4. Задачи 9-12 сгруппированы как подзадачи задачи 8 (рисунок~\ref{img:rk_t3_group}).
\imgw{rk_t3_group}{ht!}{0.7\textwidth}{Структурирование списка задач}

\section{Установление связей между задачами}

Созданы связи в соответствии с таблицей (рисунок~\ref{img:rk_t4_links}).
\imgw{rk_t4_links}{ht!}{0.7\textwidth}{Установление связей между задачами}

\newpage
\section{Создание списка ресурсов}

Создан список ресурсов (рисунок~\ref{img:rk_t5_resourses}).
\imgw{rk_t5_resourses}{ht!}{0.7\textwidth}{Список ресурсов}

Для Соколова (рисунок~\ref{img:rk_t5_note_s}) и Петрова (рисунок~\ref{img:rk_t5_note_p}) созданы заметки.
\imgw{rk_t5_note_s}{ht!}{0.7\textwidth}{Заметка для Соколова}
\imgw{rk_t5_note_p}{ht!}{0.7\textwidth}{Заметка для Петрова}

\section{Назначение ресурсов задачам}

Задачам назначены ресурсы. Также заданы фиксированные затраты: задачам 5 и 6 по 500 рублей, задаче 10 --- 1000 рублей (рисунок~\ref{img:rk_t6_gant_resourses}).
\imgw{rk_t6_gant_resourses}{ht!}{0.9\textwidth}{Назначение ресурсов задачам}

\section{Оптимизация загрузки ресурсов. Разбиение задач на подзадачи}

Разбиты на подзадачи каждая из задач по разработке представительских сайтов. Перед этим с них сняты ресурсы и удалены связи. Затем назначены на разработку дизайна Кузнецов, а на программирование сайта --- Тимофеев. Установлены связи в соответствии с таблицей (рисунок~\ref{img:rk_t8_t3_0}).
\imgw{rk_t8_t3_0}{ht!}{0.7\textwidth}{Разбиение задач на подзадачи}

Стоимость проекта --- 722 183,33 рублей. Дата окончания --- 29.06.23

\newpage
\section{Переназначение ресурсов}

Ресурс Соколов назначен задаче <<Обучение персонала>> (рисунок~\ref{img:rk_t8_t1}).
\imgw{rk_t8_t1}{ht!}{0.8\textwidth}{Назначение Соколова на <<Обучение персонала>>}

Количество единиц ресурса Преподаватели сокращено до 300 \% (рисунок~\ref{img:rk_t8_t2}).
\imgw{rk_t8_t2}{ht!}{0.9\textwidth}{Сокращение ресурса Преподаватели}

Соколов поставлен на задачу <<Обучение персонала>> после того, как он завершит работы по установке ПО (рисунок~\ref{img:rk_t8_t3_res}). Дата окончания изменилась на 23.06.23
\imgw{rk_t8_t3_res}{ht!}{0.9\textwidth}{Изменение нагрузки Соколова}

Оптимизация загрузки Соколова. Для задачи <<Установка ПО>> установлен профиль <<Загрузка в начале>>, для задачи <<Обучение персонала>> --- <<Загрузка в конце>> (рисунок~\ref{img:rk_t8_t4_gant}).
\imgw{rk_t8_t4}{ht!}{0.9\textwidth}{Изменение профиля загрузки}

\imgw{rk_t8_t4_gant}{ht!}{0.9\textwidth}{Сокращение ресурса Преподаватели}

После переназначения ресурсов стоимость проекта составила 709 350~рублей. Дата окончания --- 07.07.23. Длительность проекта увеличилась, стоимость уменьшилась.

\section{Уточнение плана проекта}

Были выделены критические задачи (рисунок~\ref{img:rk_t8_t4_gant}). Для оптимизации критического пути было принято решение назначить Тимофеева и на программирование сайта (рисунок~\ref{img:rk_t9_t}). 
\imgw{rk_t9_t}{ht!}{0.9\textwidth}{Оптимизация критического пути}

После этого стоимость проекта составила 708 100 рублей (уменьшилась на 1 250 рублей), а дата окончания сдвинулась на 30.06.23 (длительность проекта сократилась на неделю).

Затем на тестирование была назначена Фомина (рисунок~\ref{img:rk_t9_t2}). 
\imgw{rk_t9_t2}{ht!}{0.9\textwidth}{Оптимизация критического пути}

В результате стоимость проекта --- 706 016,67 рублей. Дата окончания --- 29.06.23.

По сравнению с тем, что было до оптимизации критического пути, стоимость проекта уменьшилась на 3 333,33, но все равно превысила бюджет на 59 350 рублей, а длительность проекта уменьшилась на 8 дней и оказалось в рамках календарного плана.

\section{Контроль за реализацией поекта}

Установлена дата отчета --- 07.06.23 (рисунок~\ref{img:rk_t10_t_date}). 
\imgw{rk_t10_t_date}{ht!}{0.6\textwidth}{Дата отчета}

Отклонений от базового плана нет (рисунок~\ref{img:rk_t10_t2_gant}). 
\imgw{rk_t10_t2_gant}{ht!}{0.9\textwidth}{Линия прогресса}

Меняем процент выполнения задачи программирование сайта 1, чтобы проанализировать отклонение (рисунок~\ref{img:rk_t10_t2_per_40_to_20}). 
\imgw{rk_t10_t2_per_40_to_20}{ht!}{0.9\textwidth}{Процент выполнения задачи}

После этого наблюдается отклонение (запаздывание)
(рисунок~\ref{img:rk_t10_t3_res_gant}). 
\imgw{rk_t10_t3_res_gant}{ht!}{0.9\textwidth}{Отклонение}

%\clearpage
Выбираем таблицу освоенного объема (рисунок~\ref{img:rk_t10_tabl_choose}). 
\imgw{rk_t10_tabl_choose}{ht!}{0.6\textwidth}{Выбор таблицы освоенного объема}

Таблица освоенного объема приведена на рисунке~\ref{img:rk_t10_tabl}). 
\imgw{rk_t10_tabl}{ht!}{0.9\textwidth}{Таблица освоенного объема}

На 07.06.23 (с учетом изменения процента выполнения задачи для анализа отклонений), видно, что:
\begin{itemize}[label = ---]
	\item Затраты по базовому плану (БПЗ) – 706 616,67 рублей.
	\item Отклонение от календарного плана (ОКП) < 0, проект отстает от 
	плана из-за добавленного отклонения к задаче.
	\item Отклонение по стоимости (ОПС) > 0, проект вписывается в
	смету.
	\item Отклонение по завершению (ОПЗ) > 0, нет перерасхода 
	средств.
\end{itemize}


%\section{Актуализация параметров проекта}
%
%Задана дата отчета (рисунок~\ref{img:task4_1_report_date}).
%\imgw{task4_1_report_date}{ht!}{0.7\textwidth}{Дата отчета}
%
%6 задаче установлена фактическая дата завершения (рисунок~\ref{img:task4_1_task6_res}).
%\imgw{task4_1_task6_res}{ht!}{0.7\textwidth}{Установка фактической даты завершения 6 задачи}
%
%\newpage
%Добавлен ресурс <<Специализированное ПО>> (рисунки~\ref{img:task4_1_2_po_add}-\ref{img:task4_1_2_task17_cost}).
%\imgw{task4_1_2_po_add}{ht!}{0.7\textwidth}{Материальный ресурс <<Специализированное ПО>>}
%\imgw{task4_1_2_task17_cost}{ht!}{0.7\textwidth}{Ресурсы задачи «Создание мультимедиа наполнения»}
%
%С 10 апреля на 10\% была увеличена зарплата мультимедиа-корреспондента (рисунок~\ref{img:task4_1_3_salary}).
%\imgw{task4_1_3_salary}{ht!}{0.8\textwidth}{Зарплата мультимедиа-корреспондента}
%
%Замена совещаний на презентации с 3 апреля (рисунок~\ref{img:task4_4_ret_task_resourses}).
%\imgw{task4_4_ret_task_resourses}{ht!}{0.8\textwidth}{Замена совещаний на презентации}
%
%Было произведено автоматическое выравниевание ресурсов, перегрузки устранены.
%Ресурсам презентаций также был установлен план затрат B,
%стоимость презентаций уменьшилась (рисунок~\ref{img:task4_4_ret_task_after_align}).
%\imgw{task4_4_ret_task_after_align}{ht!}{0.8\textwidth}{Стоимость презентаций с планом затрат B}
%
%С 13 марта на 5\% увеличилась стоимость аренды сервера (рисунок~\ref{img:task4_5_salary}), 
%а общая сумма проекта выросла до 50 805,88 рублей (рисунок~\ref{img:task4_5_gen_cost})
%
%\imgw{task4_5_salary}{ht!}{0.8\textwidth}{Стоимость аренды серевера}
%\imgw{task4_5_gen_cost}{ht!}{0.8\textwidth}{Стоимость проекта}
%
%По результатам актуализации параметров проекта срок проекта не был превышен, а стоимость проекта превысила запланированный бюджет.
%Для оптимизации затрат проекта были сокращены затраты на презентации посредством исключения из презентаций сотрудников, которые уже 
%окончили свои работы на момент проведения презентации: системного аналитика --- с 17.04.23, мультимедиа-корреспондента --- с 13.06.23 
%(рисунок~\ref{img:task4_6_ret_cost_1}).
%\imgw{task4_6_ret_cost_1}{ht!}{0.8\textwidth}{Стоимость презентаций после исключения из последующих презентаций сотрудников, окончивших все свои работы}
%
%Далее были добавлены 2 программиста (стало 6) на задачи,  связанные с программированием, для сокращения трудозатрати затрат соответвенно,
%т.к. они  являются высокооплачиваемыми специалистами. Это помогло ускорить завершение проекта, но не сильно уменьшило стоимость проекта.
%Т.к. сервер арендуется на время выполнения <<Построения базы объектов>>, а наборщики выполняют последнюю из ее подзадач, увеличение числа
%наборщиков данных на 2 (стало 7) позволило приблизить дату окончания проекта. Т.к. наборщики являются низкооплачиваемыми специалистами, а 
%аренда сервера на время их работы стоит дорого, сократив время работы наборщиков путем их увеличения, стоимость проекта уменьшилась и больше 
%не превышает запланированный бюджет (рисунок~\ref{img:task4_7_gen_cost}). Распредение задач между программистами и наборщиками показаны 
%на рисунках \ref{img:task4_7_view_1}-\ref{img:task4_7_view_2}.
%
%\imgw{task4_7_gen_cost}{ht!}{0.7\textwidth}{Стоимость проекта}
%
%\imgw{task4_7_view_1}{ht!}{0.7\textwidth}{Визуальный оптимизатор (часть 1)}
%\imgw{task4_7_view_2}{ht!}{0.7\textwidth}{Визуальный оптимизатор (часть 2)}
%
%
%Проект обновлен (для задач отмечен процент завершения --- рисунок~\ref{img:task4_9_upd_poject}).
%\imgw{task4_9_upd_poject}{ht!}{1\textwidth}{Настройки обновления проекта}
%
%На экран выведена линия прогресса (рисунок~\ref{img:task4_9_lines}). По ней видно, что проект не отклонился от графика.
%\imgw{task4_9_lines}{ht!}{1\textwidth}{Линия прогресса}
%
%Таким образом, стоимость проекта составила \textbf{49 132,40 рубля} (укладывается в бюджет), а дата окончания проекта --- \textbf{05.07.23} (сдвинулась на 15 дней, 
%укладывается в срок).
%
%
%\section{Работа с таблицей освоенного объема}
%
%\imgw{task5_1_cost}{ht!}{1\textwidth}{Таблица затрат}
%
%\imgw{task5_1_v_p1}{ht!}{1\textwidth}{Таблица освоенного объема (часть 1)}
%\imgw{task5_1_v_p2}{ht!}{1\textwidth}{Таблица освоенного объема (часть 2)}
%
%Видно, что ОКП положительное, значит, проект не выбился из графика по сроку. ОПС положительное, значит, 
%есть запасы бюджета на текущий момент. ОПЗ положительное, значит, нет перерасхода. Расход происходит в пределах норм. Такие показатели связаны с добавлением в проект двух программистов (высокооплачиваемые специалисты) и двух наборщиков данных (низкооплачиваемые специалисты; их увеличение уменьшило время построения базы данных и, таким образом, уменьшило затраты на сервер).
%
%\newpage
%\section{Работа с отчетами проекта}
%
%\imgw{task5_3_cost_hist}{ht!}{0.6\textwidth}{Отчет бюджетной стоимости (гистограмма)}
%\imgw{task5_3_cost_table}{ht!}{0.6\textwidth}{Отчет бюджетной стоимости}
%
%Самый дорогой --- второй квартал.
%
%\imgw{task5_3_tasks}{ht!}{0.6\textwidth}{Отчет бюджетной стоимости (гистограмма)}
%\imgw{task5_3_tasks_table}{ht!}{0.6\textwidth}{Отчет бюджетной стоимости}
%
%\imgw{task5_3_resourses}{ht!}{0.6\textwidth}{Отчет бюджетной стоимости (гистограмма)}
%\imgw{task5_3_resourses_table}{ht!}{0.6\textwidth}{Отчет бюджетной стоимости}
%
%\clearpage
%\section{Анализ вариантов декомпозиции работ в проекте}
%
%Декомпозиция по процессам и оптимизация критического пути позволила уменьшить срок проекта (новая дата окончания --- 20.07.23), 
%но при этом увеличилась стоисть проекта до 44 954 рублей. 
%
%\imgw{task_2_before}{ht!}{1\textwidth}{До декомпозиции (лр2)}
%\imgw{task5_dec_2007_gant}{ht!}{1\textwidth}{Результат декомпозиции и оптимизации критического пути}
%
%\imgw{task3_3_gen_cost_gant}{ht!}{1\textwidth}{Результат лр3}
%
%\newpage
%Сравнивая с результатами лр3, дата окончания проекта совпадает, а стоимость проекта в лр3 (48 480,18) больше, чем при декомпозиции (44 954).
%При этом стоимость совещаний, которых не было в лр3, 1769 рублей. Т.е. даже без учета совещаний разница остается в 1 757,18 рублей. Т.е. удалось сократить стоимость проекта 
%и достичь той же даты его окончания, что и в лр3.




% Информация о трудозатратах по группам ресурсов представлена в 
% графическом виде для проекта на моменты его конечных состояний из ЛР №2 и ЛР №3 (рисунки~\ref{img:task2_3_group_laborcost}-\ref{img:task3_3_group_laborcost}).

% \imgw{task2_3_group_laborcost}{ht!}{0.7\textwidth}{Информация о трудозатратах (ЛР №2)}
% \imgw{task3_3_group_laborcost}{ht!}{0.7\textwidth}{Информация о трудозатратах (ЛР №3)}

% \newpage
% По результатам добавления в план проекта совещаний, ликвидации перегрузки ресурсов, оптимизации затрат и критического пути трудозатраты изменились для следующих групп:
% \begin{itemize}[label = ---]
% 	\item Нет назначения (сервер) --- уменьшились на 2\%;
% 	\item Анализ --- увеличились на 1\%;
% 	\item Мульти-медиа --- увеличились на 1\%.
% \end{itemize}

% Информация о затратах по группам ресурсов представлена для проекта на моменты его конечных состояний из ЛР №2 и ЛР №3 (рисунки~\ref{img:task2_3_group_cost}-\ref{img:task3_3_group_cost}).

% \imgw{task2_3_group_cost}{ht!}{0.7\textwidth}{Информация о затратах (ЛР №2)}
% \imgw{task3_3_group_cost}{ht!}{0.7\textwidth}{Информация о затратах (ЛР №3)}

% \newpage
% По результатам добавления в план проекта совещаний, ликвидации перегрузки ресурсов, оптимизации затрат и критического пути затраты изменились для следующих групп:
% \begin{itemize}[label = ---]
% 	\item Нет назначения (сервер) --- уменьшились на 1\%;
% 	\item Программирование --- уменьшились на 1\%;
% 	\item Анализ --- увеличились на 1\%;
% 	\item Документация --- увеличились на 1\%.
% \end{itemize}

% \newpage
% По результатам выполнения \textbf{ЛР №2} получились следующие соотношения \textbf{<<Затраты---Трудозатраты>>}:
% \begin{itemize}[label = ---]
% 	\item Анализ --- $\frac{10}{2}=5$;
% 	\item Программирование --- $\frac{50}{29}=1.72$;
% 	\item Нет назначения (сервер) --- $\frac{13}{32}=0.4$;
% 	\item Ввод данных --- $\frac{11}{26}=0.42$;
% 	\item Документация --- $\frac{2}{2}=1$.
% \end{itemize}

% По результатам выполнения \textbf{ЛР №3} получились следующие соотношения \textbf{<<Затраты---Трудозатраты>>}:
% \begin{itemize}[label = ---]
% 	\item Анализ --- $\frac{11}{3}=3.67$ (уменьшилось);
% 	\item Программирование --- $\frac{49}{29}=1.69$ (уменьшилось);
% 	\item Нет назначения (сервер) --- $\frac{12}{30}=0.4$;
% 	\item Ввод данных --- $\frac{11}{26}=0.42$;
% 	\item Документация --- $\frac{3}{2}=1.5$ (увеличилось).
% \end{itemize}

% Сохранение \textbf{базового плана} проекта приведено на рисунке \ref{img:task3_3_base_plan}. \imgw{task3_3_base_plan}{ht!}{0.5\textwidth}{Базовый план проекта}

% \subsection*{Вывод}
% Таким образом, соотношение \textbf{<<Затраты---Трудозатраты>>} уменьшилось для высокооплачиваемых специалистов (программистов) и самых высокооплачиваемых специалистов (аналитики), но увеличилось для низкооплачиваемого специалиста (технический писатель), что в результате привело к уменьшению стоимости проекта.

%\section{Вывод по работе}
%
%По результатам выполнения ЛР №2 программисты являются высокооплачиваемыми специалистами (на них приходится 50\% затрат и лишь 29\% трудозатрат проекта), значит, при сокращении времени их работы удастся сократить затраты проекта. В процессе выполнения ЛР №3 было выяснено, для оптимизации критического пути следует сократить время на задачи программирования. Т.к. от продолжительности этих задач зависит время начала следующих, сократив их продолжительность, удалось сократить общее время выполнения проекта. Это было реализовано посредством более равномерного перераспределения программистов на их задачи.
%
%После проведенных оптимизаций также удалось уменьшить соотношение \textbf{<<Затраты---Трудозатраты>>} для программистов (высокооплачиваемые) и аналитиков (самые высокооплачиваемые), при этом оно увеличилось для технического писателя (низкооплачиваемый). В результате удалось уменьшить затраты и продолжительность проекта.
%Дата окончания проекта сдвинулась с 25.09.23 на \textbf{20.07.23} (продолжительность всего проекта сократилась почти на 2 месяца и теперь не превышает установленного срока), а стоимость проекта уменьшилась с 49 849 до \textbf{48 480,18 рублей} (уменьшилась на 1 368,82 рублей и не превышает бюджета проекта).

